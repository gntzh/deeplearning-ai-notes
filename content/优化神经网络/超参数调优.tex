\documentclass[../../main.tex]{subfiles}
\begin{document}
\chapter{超参数调优}
% Hyper parameter tuning

\section{调试处理}
% Tuning process
超参数的调试需要大规模的实验,通常在参数范围内取随机值与组合,在效果较好的区域再进行更密集的随机值与组合,由粗糙到精细。


但超参数的随机要选择合适的范围,不是所有的超参数都可以直接使用均匀分布来随机,网络层数、隐藏单元个数之类的超参数可以直接随机,但学习率这样的超参数就不该直接随机了,因为在不同数量级,学习率的敏感度是不一样的,比如在\((0.0001, 0.1)\)明显更敏感,但如果直接随机,在\((0.0001, 0.1)\)范围内的个数只有\((0.1, 1)\)的\(\frac{1}{9}\),这时可以使用对数标尺搜索超参数。

比如学习率的取值范围时\(10^{-4}, 1\),那么取10的对数,在\(-4,0\)间使用均匀分布来随机出\(r\),然后\(α=10^{r}\)。这样在不同数量级内的取值数量是一致的。

类似的还有 Momentum中的超参数\(β\),取值范围是\(0.9, 0.9999\),可以先转换成\(1-β\),此时取值范围是\((10^{-4}, 10^{-1})\),再转化为\(-4, -1\)间的均匀随机。

\section{为超参数选择合适的范围}
% Using an appropriate scale to pick hyper parameters)

\section{超参数调试的实践:Pandas VS Caviar}
% Hyper parameters tuning in practice: Pandas vs. Caviar

\section{归一化网络的激活函数}
% Normalizing activations in a network
归一化输入特征可以加快学习过程,如果对每一层都做归一化,也可以加快学习过程,在深层网络中更是作用重大。这一方法称为 Batch Normalization,算法如下:

\begin{algorithm}[H]
    \KwIn{\(B={z^{(1)}, ⋯, z^{(m)}}\); \(γ, β\)}
    \KwOut{\(\tilde{z}^{(i)} = \mathrm{BN}_{γ, β}(z_i)\)}
    \(μ_B← \frac{1}{m}\sum\limits_{i=1}^{m}z^{(i)}\)\;
    \(σ_B^2← \frac{1}{m}\sum\limits_{i=1}^{m}(z^{(i)} - μ_B)^2\)\;
    \(z_{norm}^{(i)} ← \frac{z^{(i)}-μ_B}{\sqrt{σ_B^2+ε}}\)\;
    \(\tilde{z}^{(i)} ← γz_{norm}^{(i)} + β\); \tcp{scale and shift}
\end{algorithm}
\begin{enumerate}
    \item 输入为数量为 m 的 mini-batch 在某一层线性组合输入后的\(z\)。
    \item \(γ, β\)是可训练参数,参与反向传播。
    \item 不仅有归一化,还有 scale and shift。
    \item \(ε\)除了防止分母为零的情况,还为了增加数值稳定性。
\end{enumerate}

归一化除了能减少数据的发散程度,加速神经网络训练外,缓解梯度饱和问题,sigmoid和tanh这类函数在自变量很大、很小时梯度与输出都不灵敏了,而归一化能缓解这个问题。

可训练的 scale and shift可以让神经网络自己学习、调整前面的归一化。

BN还可能减少了前后层间的依赖,缓解了协变量偏移问题 Covariate shift。简单说 Covariate shift 问题就是后层网络对前层网络依赖严重,在前层参数更新时,后层网络并不能很好地适应此时的前层网络,而BN能限制前层参数更新时对数值分布的影响,使这些值变得更稳定,后层网络更独立。

另外,BN还有一个轻微正则化的效果,因为使用的是mini-batch的\(u, σ^2\),而不是在整个数据集的,再加上缩放和偏移,这些噪声有正则化的效果。但是不应该把BN当作正则化手段,因为batch size较小的时候,正则化作用强,但同样因为\(u, σ^2\)的噪音大,导致BN的效果差(影响参数的学习),batch size调大点,正则化效果又弱了。

对z的BN使得常熟\(b\)没有意义,\(b\)是在调节均值,归一化使得\(b\)的调节失效了,可训练的\(β\)可以调节均值。

\section{将 Batch Norm 拟合进神经网络}
% Fitting Batch Norm into a neural network

\section{测试时的 Batch Norm}
% Batch Norm at test time
预测时,\(u, σ^2\)应该使用训练集所有批次的,或者训练集批次的指数加权平均值。

\section{Softmax 回归}
% Softmax regression

\section{训练一个 Softmax 分类器}
% Training a Softmax classifier

\section{深度学习框架}
% Deep Learning frameworks

\section{TensorFlow}

\end{document}