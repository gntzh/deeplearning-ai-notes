\documentclass[../../main.tex]{subfiles}
\begin{document}
\chapter{优化算法}
% Optimization algorithms

\section{Mini-Batch 梯度下降}
% Mini-Batch gradient descent

样本量很大时,一次性计算可能有问题,这时可以将样本随机分成若干 mini-batch,\(X^{\{t\}}, y^{\{t\}}\)表示第\(t\)个 mini-batch。当 batch size 取值不同时:
\begin{itemize}
    \item m:Batch Gradient Descent,最为稳定,但样本量大时不可用。
    \item 1:Stochastic Gradient Descent,噪音太大,波动前进,基本不会收敛。
    \item (1, m):Mini-Batch Gradient Descent。
\end{itemize}
\begin{remark}
    现在一般提起SGD是指Mini-Batch Gradient Descent。
\end{remark}


\section{理解 Mini-Batch 梯度下降}
% Understanding mini-batch gradient descent

\section{指数加权平均数}
% Exponentially weighted averages
% TODO 补充,指数加权平均数的推导
对于有序数据,比如每日气温,如果想要不保存前几日的气温而求前几日与今日的平均气温,可以使用\textbf{指数加权平均数 Exponentially weighted averages}:
\[v_t = β*v_{t-1} + (1-β)θ_t ≈ (1-β)\sum\limits_{i=t + 1 - \frac{1}{1-β}}^{t}v_i\]
比如\(β=0.1, \frac{1}{1-β}=10\),则可近似计算前10日的平均气温。


\section{理解指数加权平均数}
% Understanding exponentially weighted averages

\section{指数加权平均的偏差修正}
% Bias correction in exponentially weighted averages
因为设\(v_0 = 0\),导致前期偏差过大,可以使用\(\frac{v_t}{1-β^t}\)来代替,以修正偏差。

\section{动量梯度下降法}
% Gradient descent with Momentum
使用指数加权平均数可以简单的保存平均梯度,用以修正当前的梯度:
\begin{align*}
     & v_t = βv_{t-1} + (1-β)g_{t} \\
     & p_t = p_{t-1} - αv_t
\end{align*}
其中,\(v_t\)表示平均梯度,即 Momentum。采用动量 Momentum 后,在某一维度上,当梯度方向不变时,更新速度变快,当梯度方向有所改变时,更新速度变慢,从而加快收敛速度,减少震荡。平均梯度就类似动量,当前梯度类似冲量。优点:
\begin{itemize}
    \item 梯度更新除了受同一学习率影响外,还有每个参数各自的动量。
    \item 在梯度改变方向的时候减少震荡,在梯度方向一致时加速收敛。
    \item 可能能够跳出局部最优陷阱。
\end{itemize}

另外还有其他形式,比如PyTorch采用的:
\begin{align*}
     & v_t = βv_{t-1} + g_{t} \\
     & p_t = p_{t-1} - αv_t
\end{align*}
其他框架采用的:
\begin{align*}
     & v_t = βv_{t-1} + αg_{t} \\
     & p_t = p_{t-1} - v_t
\end{align*}

\begin{remark}
    Momentum 的取值可以考虑0.9,这是一个 Robust Value。
\end{remark}

这里有一篇文章 \href{https://distill.pub/2017/momentum/}{Why Momentum Really Works},解释了动量梯度下降法为什么有效,还做了可视化。
% TODO 补充,Nesterov Accelerated Gradient(NAG)

\section{RMSprop}
% RMSprop
\textbf{RMSprop} 全称 root mean square prop,相比 Momentum 使用的是二阶矩:
\begin{align*}
     & S_t = βv_{t-1} + (1-β)g_{t}^2        \\
     & p_t = p_{t-1} - \frac{α}{\sqrt{S_t}}
\end{align*}
% TODO

\section{Adam 优化算法}
% Adam optimization algorithm
\textbf{Adam},全称Adaptive Moment Estimation,结合了 Momentum 和 RMSprop,并做了偏差修正:
\begin{align*}
     & v_t = β_1v_{t-1} + (1-β_1)g_{t}                                     \\
     & S_t = β_2v_{t-1} + (1-β_2)g_{t}^2                                   \\
     & v_t^{corrected} = \frac{v_t}{1-β_1^t}                               \\
     & S_t^{corrected} = \frac{S_t}{1-β_2^t}                               \\
     & p_t = p_{t-1} - α\frac{v_t^{corrected}}{\sqrt{S_t^{corrected}} + ε}
\end{align*}
Adam 中有四个超参数,其中学习率\(α\)是经常需要调试的,\(ε\)不会影响算法表现,使用默认值\(10^{-8}\)即可,\(β_1\)和\(β_2\)默认值为\(0.9, 0.999\)。

\section{学习率衰减}
% Learning rate decay
随着迭代次数增加,学习率应该衰减,尤其在最小值附近,如果学习率不变,会大幅度在最小值附近摆动。

有时还需要手动调整学习率。

\section{局部最优的问题}
% The problem of local optima
在高维空间中,不太可能困在极差的局部最优中,更可能遭遇平稳段 plateau 或被困在鞍点 saddle point,这使得学习十分缓慢,只能靠优化算法或手动调整相关参数,(更快)走出平稳段

\end{document}