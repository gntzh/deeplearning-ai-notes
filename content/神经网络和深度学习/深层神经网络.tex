\documentclass[../../main.tex]{subfiles}
\begin{document}
\chapter{深层神经网络}
% Deep Neural Networks
\section{L层深层神经网络}
% Deep L-layer neural network
规定一些符号:
\begin{itemize}
    \item \(L\)表示神经网络的层数(不含输入层)
    \item \(n^{[l]}\)表示第\(l\)层激活单元的个数
    \item \(b^{[l]}\)表示第\(l\)层的偏置矩阵
    \item \(W^{[l]}\)表示第\(l\)层的权重矩阵
    \item \(g^{[l]}\)表示第\(l\)层的激活函数
    \item \(z^{[l]}\)表示第\(l\)层的输入
    \item \(a^{[l]}\)表示第\(l\)层的输出
    \item \(\hat{y}\)表示预测,\(\hat{y}=a^{[l]}\)
\end{itemize}
\section{前向传播和反向传播}
% Forward and backward propagation

\section{深层网络中的前向传播}
% Forward propagation in a Deep Network

\section{核对矩阵的维数}
% Getting your matrix dimensions right

\section{为什么使用深层表示?}
% Why deep representations?
可以从两个方面理解:
\begin{enumerate}
    \item 深层网络从小到大挖掘特征组合,从而发挥作用。\\更多的层数才能挖掘到更多、更整体的组合
    \item 深层网络类似电路,自然电路门越多越强大。\\而且层数太小所需要的元件(神经元)更多,例如计算n个数的异或,如果仅有一层则需要\(2^{n-1}\)个神经元,如果使用深度网络,需要\(\log(n)\)层深度。
\end{enumerate}
\section{搭建神经网络块}
% Building blocks of deep neural networks

\section{参数VS超参数}
% Parameters vs Hyper Parameters
简言之,\textbf{参数}是根据数据训练得出的,\textbf{超参数}是手动设置或启发式算法设置的。

参数具有这些特点:
\begin{itemize}
    \item 模型在进行预测时需要它们。
    \item 它们的值定义了可使用的模型。
    \item 它们是从数据估计或获悉的。
    \item 它们通常不由编程者手动设置。
    \item 它们通常被保存为学习模型的一部分。
\end{itemize}
常见的参数有NN中的权重、SVM中的支持向量、线性回归或逻辑回归中的系数。

超参数具有这些特点:
\begin{itemize}
    \item 它们通常用于帮助估计模型参数。
    \item 它们通常由人工指定。
    \item 它们可以使用启发式设置。
    \item 它们经常被调整为给定的预测建模问题。
\end{itemize}
常见的超参数有学习速率、SGD中的momentum、正则化参数。

我们无法知道给定问题的超参数的最佳值,只能用经验法则和通过反复试验来寻找最佳值。

\section{这和大脑一定有关吗?}
% What does this have to do with the brain?
深度学习或者说神经网络确实来自对大脑的模仿,但仍与大脑关联不大,人类大脑是如此强大和复杂,目前连单个神经元也没有搞清楚。
\end{document}